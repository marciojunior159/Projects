% Relatório da versão 1 do software ipump para o curso
% Sistemas de Controle - DCA0206 - UFRN
% Autores:
%   AUGUSTO MATHEUS PINHEIRO DAMASCENO
%   MARCEL DA CÂMARA RIBEIRO DANTAS
%   PABLO HOLANDA CARDOSO
%   PEDRO DE CASTRO GURGEL LIMA
%   RODRIGO DANTAS DA SILVA
% Modificado por: Ícaro Bezerra Queiroz de Araújo
%

%%%%%%%%%%%% STRUCTURE %%%%%%%%%%%%%%%
\documentclass[a4paper,12pt]{article}
\usepackage[T1]{fontenc}
\usepackage[utf8]{inputenc}
\usepackage[brazil]{babel}
\usepackage{lmodern}
\usepackage{setspace}
\usepackage[top=2cm, bottom=2cm, left=2cm, right=2cm]{geometry}
%%%%%%%%%%%%%%%%%%%%%%%%%%%%%%%%%%%%%%

%%%%%%%%%%%%%%%% PAGES STYLE %%%%%%%%%
\usepackage{fancyhdr}
\fancypagestyle{main}{
\renewcommand{\headrulewidth}{0pt}
\fancyhead[RO]{\thepage}
\fancyfoot[CO]{}
}
%%%%%%%%%%%%%%%%%%%%%%%%%%%%%%%%%%%%%%

\usepackage{graphicx}
\usepackage{float}
\usepackage{epstopdf}
\usepackage{subfig}
\usepackage{mathptmx}
\usepackage{changepage}


\usepackage{listings}
\usepackage{xcolor}
\lstset{language=C++,
                basicstyle=\ttfamily,
                keywordstyle=\color{blue}\ttfamily,
                stringstyle=\color{red}\ttfamily,
                commentstyle=\color{green}\ttfamily,
                morecomment=[l][\color{magenta}]{\#}
}

%\usepackage[alf]{abntex2cite}

%%%%%%%%%%% PDF METADATA %%%%%%%%%%%%%
\usepackage[ pdftitle={MODELO RELATÓRIO},
pdfsubject={INTRODUÇÃO AO LABORATÓRIO DE CONTROLE - Grupo 3},
pdfkeywords={Controle,Automação,UFRN,DCA,ipump},
hidelinks]{hyperref}
%%%%%%%%%%%%%%%%%%%%%%%%%%%%%%%%%%%%%%

\begin{document}

\onehalfspacing

\thispagestyle{empty}

\setcounter{page}{1}

%%%%%%%%%%%% LOGOS %%%%%%%%%%%%%%%%%%%

\begin{figure}[!ht]

\centering

\subfloat{
\includegraphics[width=2.7cm]{UFRN.eps}
\label{UFRN Logo}
}
\hspace{11.09cm}
\subfloat{
\includegraphics[width=2.4cm]{DCA.eps}
\label{DCA Logo}
}

%\caption{}
\label{Logos}

\end{figure}

%%%%%%%%%%%%%%% CAPA %%%%%%%%%%%%%%%%%

\vspace{-1cm}

\begin{center}
{\bf{\normalsize UNIVERSIDADE FEDERAL DO RIO GRANDE DO NORTE\\
CENTRO DE TECNOLOGIA\\
DEPARTAMENTO DE ENGENHARIA DE COMPUTAÇÃO E AUTOMAÇÃO\\
CURSO DE ENGENHARIA DE COMPUTAÇÃO
}}


\vspace{3.6cm}

{\bf{\large RELATÓRIO DA 1ª EXPERIÊNCIA\\
INTRODUÇÃO AO LABORATÓRIO DE CONTROLE\\
}}
\vspace{1.5cm}
{\large TURMA: 01 A\\
	GRUPO Nº}

\vspace{3.6cm}


\begin{flushright}
\begin{normalsize}
ANDRESSA STÉFANY SILVA DE OLIVEIRA: 20160154101\\
\vspace{0.8cm}
FERNANDA MONTEIRO DE ALMEIDA: 20160154228\\
\vspace{0.8cm}
VITOR RAMOS GOMES DA SILVA: 20160154415\\
\vspace{0.8cm}
MÁRCIO LUIZ BEZERRA LOPES JÚNIOR: 20160154326\\
\end{normalsize}
\end{flushright}


\vspace{2.5cm}

{\large Natal-RN\\
2017}

\end{center}

\newpage

%%%%%%%%%%%%%%%  CONTRA-CAPA %%%%%%%%%

\thispagestyle{empty}

\begin{center}
\begin{normalsize}
ANDRESSA STÉFANY SILVA DE OLIVEIRA: 2016015410\\
\vspace{0.8cm}
FERNANDA MONTEIRO DE ALMEIDA 20160154228\\
\vspace{0.8cm}
VITOR RAMOS GOMES DA SILVA: 20160154415\\
\vspace{0.8cm}
MÁRCIO LUIZ BEZERRA LOPES JÚNIOR: 20160154326\\

\end{normalsize}
\end{center}
\vspace{3cm}

{\bf{\large {\centering INTRODUÇÃO AO LABORATÓRIO DE CONTROLE\\}}}

\vspace{4cm}

\begin{adjustwidth}{7.5cm}{0cm}

{\normalsize

Primeiro Relatório Parcial apresentado à disciplina de
Laboratório de Sistemas de Controle, correspondente à
avaliação da 1º unidade do semestre 2017.1 do 8º período
do curso de Engenharia de Computação da
Universidade Federal do Rio Grande do Norte, sob
orientação do {\bf Prof. Fábio Meneghetti Ugulino de
Araújo.}

}

\end{adjustwidth}

\vspace{2cm}

\begin{center}

Professor:  Fábio Meneghetti Ugulino de Araújo.

\vspace{2.5cm}

{\large Natal-RN\\
2017}

\end{center}

\newpage

%%%%%%%%%%%%%%%  RESUMO %%%%%%%%%%%%%%

\thispagestyle{empty}

\begin{center}
{\large \textbf{RESUMO}}
\end{center}

\vspace{3cm}

\begin{flushleft}

\hspace{4ex}O presente trabalho é referente ao desenvolvimento de um software que se comunica com um sistema de tanques, uma planta Quanser, e seu simulador, o qual possui canais para leitura e escrita de sinais. Primeiro, houve a análise matemática das relações entre vazão de entrada, vazão de saída, tensão enviada para a bomba e o nível de água presente no tanque. Além disso, o sistema possui duas situações: malha aberta, a qual o usuário escolhe a tensão a ser enviada; e malha fechada, nesse caso, o usuário indica exatamente o nível que a água deve estar. O usuário também escolhe o tipo de sinal que está sendo enviado (degrau, onda quadrada, entre outros). Posteriormente, fazendo uso de dados adquiridos experimentalmente, obteve as equações referente ao comportamento do fluxo da água com relação à bomba, como também, a associação da altura com a tensão. O comportamento do sistema é informado para o usuário através de gráficos presentes na interface do software, tanto o sinal de entrada escolhido como os valores que são lidos nos canais indicados pelo usuário.\\

\end{flushleft}

\vspace{1.5cm}

\textbf{Palavras-chave:} sistema de tanques; sistema de controle; software; planta Quanser.

\newpage

%%%%%%%%% LISTA DE FIGURAS %%%%%%%%%%%

\thispagestyle{empty}

\begin{center}
\listoffigures
\end{center}

\newpage

%%%%%%%%%%%%%%% SUMÁRIO %%%%%%%%%%%%%%

\thispagestyle{empty}

\begin{center}
\tableofcontents
\end{center}

\newpage

%%%%%%%%%%%%%%% INTRODUÇÃO %%%%%%%%%%%

\thispagestyle{main}

\section{INTRODUÇÃO}

\begin{flushleft}
\hspace{4ex}O trabalho descrito neste relatório é uma continuação do programa desenvolvido na primeira experîência.

\hspace{4ex}Além do usuário possuir a opção de malha aberta e malha fechada, onde antes era apenas informado o sinal e a tensão ou nível desejado, agora é possível escolher entre os controladores Proporcional(P), Proporcional Integral(PI), Proporcional Derivativo(PD), Proporcional Integral Derivativo(PID) e (Proporcional+Integral)+Derivativo(PI-D), e assim, de acordo com o controlador escolhido, a determinação de Kp, Ki e Kd, ou    $\tau_i$ e $\tau_d.$

\hspace{4ex}Além disso, o relatório dispõe de um referêncial teórico 

 tem como objetivo o desenvolvimento de uma aplicação capaz de manipular um sistema de tanques Quanser através da utilização da linguagem C++.

A aplicação deve ser capaz de controlar o nível no tanque superior de acordo com a orientação dada pelo usuário. O usuário define a tensão da bomba em volts (para o caso de malha aberta) ou o nível a ser atingido no tanque em centímetros (para o caso de malha fechada), sendo estabelecidos limites na aplicação respeitando os limites físicos do tanque e da bomba. A aplicação deverá realizar esse controle utilizando-se de cinco tipos de sinais: degrau, onda senoidal, onda quadrada, onda do tipo dente de serra, e aleatória. O usuário definirá também as portas de entrada e saída, podendo escolher de uma a oito portas de saída.

Através das equações de vazão, pretende-se achar uma relação entre a tensão aplicada pela bomba e nível de água no tanque, de forma que o nível varie o mínimo possível. A partir dessa relação serão desenvolvidas as funções referentes aos cinco tipos de sinais.

O resultado será apresentado através de uma interface gráfica, onde o usuário, além de controlar a situação, poderá visualizar as variações ocorrentes nas portas de saída em plano cartesiano, incluindo os níveis nos dois tanques e a tensão na bomba.
\end{flushleft}

\newpage

%%%%%%%%%% REFERENCIAL TEÓRICO %%%%%%%

\thispagestyle{main}

\section{REFERENCIAL TEÓRICO}

\subsection{MODELAGEM}
\hspace{4ex}O sistema a ser controlado é representado pelo seguinte sistema.
\begin{figure}[h]
\includegraphics[scale=1]{tanques.png}
\caption{Sistema de tanques}
\label{fig:tanques}
\end{figure}

\hspace{4ex}Sabemos que a vazão de entrada é dada por $ k_mu(t) $ e a vazão de saída $ \sqrt{2gh(t)}a $, onde $k_m$ é a constante da bomba e $"a"$ a área do orificio. Assim: 
\[ \frac{dq}{dt}=qin-qout\]
\[ A\frac{dh(t)}{dt}=k_mu(t)-\sqrt{2gh(t)}a \]
\[ A\frac{dh(t)}{dt}=k_mu(t)-\sqrt{2gh(t)}a \]
\[ \frac{dh(t)}{dt}= \frac{k_m}{A}u(t)- \frac{a\sqrt{2gh(t)}}{A} \]

\subsection{ANALISE}
\hspace{4ex}Para atingir um certo nível no tanque estamos interessados no valor de tensão que fará o sistema alcançar esse nível. Para isso precisamos analisar o regime permanente. Após algum tempo a altura praticamente não vai mais variar, então podemos considerar.  $\frac{dh(t)}{dt}=0 $, com isso nossa equação fica:
\[ \frac{k_m}{A}u(t)= \frac{a\sqrt{2gh(t)}}{A} \]
\[ u(t)=  \frac{a\sqrt{2gh(t)}}{k_m}\]
\begin{equation}\label{eq:1}
u(t)=k\sqrt{h(t)}
\end{equation}
\hspace{4ex}Por questões de simplificação e como a variação da altura é pequena, aproximamos para uma função linear. Com a relação do sinal de entrada com o nível final do tanque, podemos calcular experimentalmente o valor de k, aplicando um sinal de entrada do tipo degrau e observando o seu valor final.
Com o valor de k calculado, dado um nível sabemos qual a tensão necessária para atingi-lo.

\subsection{SENSORES}
\hspace{4ex}Para obter o nível dos tanques são utilizados sensores de pressão. a saída desses sensores são uma tensão proporcional a pressão e como a pressão é proporcional a altura podemos encontrar uma relação linear da tensão com a altura da água.


\subsection{SISTEMA EM MALHA ABERTA}
\begin{figure}[H]
\includegraphics[width=15cm]{malhaAberta.png}
\caption{Malha aberta}
\label{fig:malhaAberta}
\end{figure}
\hspace{4ex}No processo em malha aberta não existe nenhum controle o sinal de entrada é a tensão e a saída o nível do tanque que é lida pelo sensor.


\subsection{SISTEMA EM MALHA FECHADA}
\hspace{4ex}No processo em malha fechada o sinal de entrada é o nível de referência.
A entrada do controlador é a diferença de nivel desejado para o medido, a saida do controlador é a própria diferença somada com a tensão necessária para o sistema estabilizar naquele nível, utilizando a equação \ref{eq:1}.
Assim a tensão somada dá o valor necessário para estabilizar e o erro acelera esse processo.
O período de amostragem e o controle do sistema é feito a cada 100 milisegundos.


\begin{figure}[H]
\includegraphics[width=15cm]{malhaFechada.jpg}
\caption{Malha Fechada}
\label{fig:malhaFechada}
\end{figure}


\newpage

%%%%%%%%%% METODOLOGIA %%%%%%%%%%%%%%%

\thispagestyle{main}

\section{METODOLOGIA}

\hspace{4ex}O sistema de controle Proporcional Integral Derivativo (PID) desenvolvido foi implementado no código de acordo com o referencial teórico. Para isso, foi criado uma classe que de acordo com a solicitação do usuário será: apenas Proporcional (P), ou Proporcional Integrativo (PI), ou Proporcional Derivativo (PD), ou PID, ou PI-D. A implementação pode ser observada abaixo:

\begin{lstlisting}
double PID::Controle(double e, double h)
{
    I= I+Ki*e*h;
    D= Kd*(e-e_ant)/h;
    e_ant= e;
    return Kp*e+I+D;
}

double PID::Controle(double e, double y, double h)
{
    I= I+Ki*e*h;
    D= Kd*(y-e_ant)/h;
    e_ant= y;
    return Kp*e+I+D;
}
\end{lstlisting}

\hspace{4ex}O primeiro método é referente ao P, PI, PD e PID, enquanto que o segundo método é o PI-D.

\hspace{4ex}Após essa implementação, foram feitos vários testes utilizando todos os tipos de controladores, como também, variando os ganhos.

\newpage

%%%%%%%%%% RESULTADOS %%%%%%%%%%%%%%%

\thispagestyle{main}

\section{RESULTADOS}

\hspace{4ex}Bla bla

\subsection{Controle P}
\hspace{4ex}Utilizando o Controle Proporcional foram obtidos vários resultados variando o valor de Kp, como também o nível desejado, observe a figura \ref{fig:ControleP}.

\hspace{4ex}Notou-se que com o aumento do ganho o sistema se aproxima da referência, mas nunca a alcança. Além disso, a partir de um certo valor de Kp, o sistema começa a ficar instável, como pode ser observado nas figuras \ref{<figureP6>} e \ref{<figureP7>}.
\begin{figure}[H]
     \centering
     \subfloat[][]{\includegraphics[width=8cm]{resultados/P/00}\label{<figureP1>}}\hspace{4ex}
     \subfloat[][]{\includegraphics[width=8cm]{resultados/P/01}\label{<figureP2>}}\\
     \subfloat[][]{\includegraphics[width=8cm]{resultados/P/02}\label{<figureP3>}}\hspace{4ex}
     \subfloat[][]{\includegraphics[width=8cm]{resultados/P/03}\label{<figureP4>}}\\
     \subfloat[][]{\includegraphics[width=8cm]{resultados/P/04}\label{<figureP5>}}\hspace{4ex}
     \subfloat[][]{\includegraphics[width=8cm]{resultados/P/05}\label{<figureP6>}}\\
     \subfloat[][]{\includegraphics[width=8cm]{resultados/P/06}\label{<figureP7>}}\hspace{4ex}
     \subfloat[][]{\includegraphics[width=8cm]{resultados/P/07}\label{<figureP8>}}
     \caption{Controle P}
     \label{fig:ControleP}
\end{figure}

\subsection{Controle PI}
\hspace{4ex}Abaixo, na figura \ref{controlePI}, é possível analisar resultados obtidos pela utilização do controle Proporcional Integral.

\hspace{4ex}Como esperado, o nível do tanque consegue atingir o sinal de referência, pois o controlado PI zera o erro de regime para uma entrada degrau.

\hspace{4ex}Também foi verificado que com o aumento do Ki houve um aumento do overshoot, isso pode ser notado nas figuras \ref{<figurePI3>} e \ref{<figurePI4>}.
\begin{figure}[H]
     \centering
     \subfloat[][]{\includegraphics[width=8cm]{resultados/PI/00}\label{<figurePI1>}}\hspace{4ex}
     \subfloat[][]{\includegraphics[width=8cm]{resultados/PI/01}\label{<figurePI2>}}\\
     \subfloat[][]{\includegraphics[width=8cm]{resultados/PI/02}\label{<figurePI3>}}\hspace{4ex}
     \subfloat[][]{\includegraphics[width=8cm]{resultados/PI/03}\label{<figurePI4>}}\\
     \subfloat[][]{\includegraphics[width=8cm]{resultados/PI/04}\label{<figurePI5>}}\hspace{4ex}
     \subfloat[][]{\includegraphics[width=8cm]{resultados/PI/05}\label{<figurePI6>}}\\
     \subfloat[][]{\includegraphics[width=8cm]{resultados/PI/06}\label{<figurePI7>}}
     \caption{Controle PI}
     \label{controlePI}
\end{figure}

\subsection{Controle PD}
\hspace{4ex}O controlador PD tenta prever variações no sinal, dessa forma, melhorando o transitório, veja a figura \ref{controlePD}.

\hspace{4ex}Como também, quando há o aumento do valor do Kd, o sinal de controle fica muito oscilatório como visto na figura \ref{<figurePD4>}.
\begin{figure}[H]
     \centering
     \subfloat[][]{\includegraphics[width=8cm]{resultados/PD/00}\label{<figurePD1>}}\hspace{4ex}
     \subfloat[][]{\includegraphics[width=8cm]{resultados/PD/01}\label{<figurePD2>}}\\
     \subfloat[][]{\includegraphics[width=8cm]{resultados/PD/02}\label{<figurePD3>}}\hspace{4ex}
     \subfloat[][]{\includegraphics[width=8cm]{resultados/PD/03}\label{<figurePD4>}}\\
     \subfloat[][]{\includegraphics[width=8cm]{resultados/PD/04}\label{<figurePD5>}}\hspace{4ex}
     \subfloat[][]{\includegraphics[width=8cm]{resultados/PD/05}\label{<figurePD6>}}\\
     \subfloat[][]{\includegraphics[width=8cm]{resultados/PD/06}\label{<figurePD7>}}
     \caption{Controle PD}
     \label{controlePD}
\end{figure}

\subsection{Controle PID}
\hspace{4ex}Usando o controlador PID é possível constatar uma diminuição do overshoot e no tempo de resposta.
\begin{figure}[H]
     \centering
     \subfloat[][]{\includegraphics[width=8cm]{resultados/PID/00}\label{<figurePID1>}}\hspace{4ex}
     \subfloat[][]{\includegraphics[width=8cm]{resultados/PID/01}\label{<figurePID2>}}\\
     \subfloat[][]{\includegraphics[width=8cm]{resultados/PID/02}\label{<figurePID3>}}\hspace{4ex}
     \subfloat[][]{\includegraphics[width=8cm]{resultados/PID/03}\label{<figurePID4>}}\\
     \subfloat[][]{\includegraphics[width=8cm]{resultados/PID/04}\label{<figurePID5>}}\hspace{4ex}
     \subfloat[][]{\includegraphics[width=8cm]{resultados/PID/05}\label{<figurePID6>}}\\
     \subfloat[][]{\includegraphics[width=8cm]{resultados/PID/06}\label{<figurePID7>}}
     \caption{Controle PID}
     \label{steady_state}
\end{figure}

\subsection{Controle PI-D}
\hspace{4ex}Fazendo uso do controlador PI-D é possível ver que existe menos variações bruscas no sinal de controle, além do tempo de resposta ser mais rápido, veja a figura \ref{controlePI-D}. Esse sinal parece ser melhor para sinais que possuem uma maior variancia.
\begin{figure}[H]
     \centering
     \subfloat[][]{\includegraphics[width=8cm]{resultados/PI-D/00}\label{<figurePI_D1>}}\hspace{4ex}
     \subfloat[][]{\includegraphics[width=8cm]{resultados/PI-D/01}\label{<figurePI_D2>}}\\
     \subfloat[][]{\includegraphics[width=8cm]{resultados/PI-D/02}\label{<figurePI_D3>}}\hspace{4ex}
     \subfloat[][]{\includegraphics[width=8cm]{resultados/PI-D/03}\label{<figurePI_D4>}}\\
     \subfloat[][]{\includegraphics[width=8cm]{resultados/PI-D/04}\label{<figurePI_D5>}}\hspace{4ex}
     \subfloat[][]{\includegraphics[width=8cm]{resultados/PI-D/05}\label{<figurePI_D6>}}\\
     \subfloat[][]{\includegraphics[width=8cm]{resultados/PI-D/06}\label{<figurePI_D7>}}\hspace{4ex}
     \subfloat[][]{\includegraphics[width=8cm]{resultados/PI-D/07}\label{<figurePI_D8>}}
     \caption{Controle PI-D}
     \label{controlePI-D}
\end{figure}

\newpage

%%%%%%%%%% CONCLUSÃO %%%%%%%%%%%%%%%

\thispagestyle{main}

\section{CONCLUSÃO}


\hspace{4ex}De acordo com os resultados obtidos, pode-se concluir que na malha fechada, o controlador consegue rastrear bem o degrau, para os demais sinais, esse rastreamento só é possível se a variação do sinal for mais lenta que a constante de tempo do sistema. Enquanto que na malha aberta, é importante ressaltar que o sinal dificilmente ficará no valor calculado teoricamente, devido a interferência de fatores externos e internos, como por exemplo a temperatura, funcionamento da bomba e ajustes dos sensores da planta.

Além disso, o controlador utilizado na malha fechada calcula o erro do nível do tanque e soma com a tensão necessária para chegar na altura desejada pelo usuário, dessa forma, o erro faz o papel de acelerador do sistema, fazendo com que um determinado nível seja alcançado rapidamente, diferentemente do caso em que se utiliza apenas o valor do erro onde o valor desejado pode não ser alcançado.

\newpage

%%%%%%%% REFERÊNCIAS %%%%%%%%%%%%%%%%%

%Referências bibliogáficas (geradas automaticamente)
\addcontentsline{toc}{chapter}{Referências bibliográficas}
\bibliography{bib/bibliografia}

\appendix

%Apêndice A
\include{apendice}

\end{document}
