% Relatório da versão 1 do software ipump para o curso
% Sistemas de Controle - DCA0206 - UFRN
% Autores:
%   AUGUSTO MATHEUS PINHEIRO DAMASCENO
%   MARCEL DA CÂMARA RIBEIRO DANTAS
%   PABLO HOLANDA CARDOSO
%   PEDRO DE CASTRO GURGEL LIMA
%   RODRIGO DANTAS DA SILVA
% Modificado por: Ícaro Bezerra Queiroz de Araújo
%

%%%%%%%%%%%% STRUCTURE %%%%%%%%%%%%%%%
\documentclass[a4paper,12pt]{article}
\usepackage[T1]{fontenc}
\usepackage[utf8]{inputenc}
\usepackage[brazil]{babel}
\usepackage{lmodern}
\usepackage{setspace}
\usepackage[top=2cm, bottom=2cm, left=2cm, right=2cm]{geometry}
%%%%%%%%%%%%%%%%%%%%%%%%%%%%%%%%%%%%%%

%%%%%%%%%%%%%%%% PAGES STYLE %%%%%%%%%
\usepackage{fancyhdr}
\fancypagestyle{main}{
\renewcommand{\headrulewidth}{0pt}
\fancyhead[RO]{\thepage}
\fancyfoot[CO]{}
}
%%%%%%%%%%%%%%%%%%%%%%%%%%%%%%%%%%%%%%

\usepackage{graphicx}
\usepackage{float}
\usepackage{epstopdf}
\usepackage{subfig}
\usepackage{mathptmx}
\usepackage{changepage}


\usepackage{listings}
\usepackage{xcolor}
\lstset{language=C++,
                basicstyle=\ttfamily,
                keywordstyle=\color{blue}\ttfamily,
                stringstyle=\color{red}\ttfamily,
                commentstyle=\color{green}\ttfamily,
                morecomment=[l][\color{magenta}]{\#}
}

%\usepackage[alf]{abntex2cite}

%%%%%%%%%%% PDF METADATA %%%%%%%%%%%%%
\usepackage[ pdftitle={MODELO RELATÓRIO},
pdfsubject={INTRODUÇÃO AO LABORATÓRIO DE CONTROLE - Grupo 3},
pdfkeywords={Controle,Automação,UFRN,DCA,ipump},
hidelinks]{hyperref}
%%%%%%%%%%%%%%%%%%%%%%%%%%%%%%%%%%%%%%

\begin{document}

\onehalfspacing

\thispagestyle{empty}

\setcounter{page}{1}

%%%%%%%%%%%% LOGOS %%%%%%%%%%%%%%%%%%%

\begin{figure}[!ht]

\centering

\subfloat{
\includegraphics[width=2.7cm]{UFRN.eps}
\label{UFRN Logo}
}
\hspace{11.09cm}
\subfloat{
\includegraphics[width=2.4cm]{DCA.eps}
\label{DCA Logo}
}

%\caption{}
\label{Logos}

\end{figure}

%%%%%%%%%%%%%%% CAPA %%%%%%%%%%%%%%%%%

\vspace{-1cm}

\begin{center}
{\bf{\normalsize UNIVERSIDADE FEDERAL DO RIO GRANDE DO NORTE\\
CENTRO DE TECNOLOGIA\\
DEPARTAMENTO DE ENGENHARIA DE COMPUTAÇÃO E AUTOMAÇÃO\\
CURSO DE ENGENHARIA DE COMPUTAÇÃO
}}


\vspace{3.6cm}

{\bf{\large RELATÓRIO DA 4ª EXPERIÊNCIA\\
CONTROLE DE SISTEMAS DINÂMICOS: CONTROLE EM CASCATA\\
}}
\vspace{1.5cm}
{\large TURMA: 01 A\\
	GRUPO Nº 02}


\vspace{3.6cm}


\begin{flushright}
\begin{normalsize}
ANDRESSA STÉFANY SILVA DE OLIVEIRA: 20160154101\\
\vspace{0.8cm}
FERNANDA MONTEIRO DE ALMEIDA: 20160154228\\
\vspace{0.8cm}
MÁRCIO LUIZ BEZERRA LOPES JÚNIOR: 20160154326\\
\vspace{0.8cm}
VITOR RAMOS GOMES DA SILVA: 20160154415\\
\end{normalsize}
\end{flushright}


\vspace{2.5cm}

{\large Natal-RN\\
2017}

\end{center}

\newpage

%%%%%%%%%%%%%%%  CONTRA-CAPA %%%%%%%%%

\thispagestyle{empty}

\begin{center}
\begin{normalsize}
ANDRESSA STÉFANY SILVA DE OLIVEIRA: 2016015410\\
\vspace{0.8cm}
FERNANDA MONTEIRO DE ALMEIDA 20160154228\\
\vspace{0.8cm}
MÁRCIO LUIZ BEZERRA LOPES JÚNIOR: 20160154326\\
\vspace{0.8cm}
VITOR RAMOS GOMES DA SILVA: 20160154415\\

\end{normalsize}
\end{center}
\vspace{3cm}

{\bf{\large {\centering CONTROLE DE SISTEMAS DINÂMICOS: CONTROLE EM CASCATA\\}}}

\vspace{4cm}

\begin{adjustwidth}{7.5cm}{0cm}

{\normalsize
Quarto Relatório apresentado à disciplina de
Laboratório de Sistemas de Controle, correspondente à
avaliação da 2º unidade do semestre 2017.1 do 8º período
do curso de Engenharia de Computação da
Universidade Federal do Rio Grande do Norte, sob
orientação do {\bf Prof. Fábio Meneghetti Ugulino de
Araújo} e {\bf Prof. Lucas Costa Pereira Cavalcante.}

}

\end{adjustwidth}

\vspace{2cm}

\begin{center}

Professores:  Fábio Meneghetti Ugulino de Araújo e\\
Lucas Costa Pereira Cavalcante.

\vspace{2.5cm}

{\large Natal-RN\\
2017}

\end{center}

\newpage

%%%%%%%%%%%%%%%  RESUMO %%%%%%%%%%%%%%

\thispagestyle{empty}

\begin{center}
{\large \textbf{RESUMO}}
\end{center}

\vspace{3cm}

\begin{flushleft}

\hspace{4ex}O presente trabalho é a terceira etapa da construção de um microcontrolador para controle de sistemas. Consiste no desenvolvimento de um software que se comunica com um sistema de tanques, uma planta Quanser, e seu simulador, com canais para leitura e escrita de sinais. Nesta etapa do trabalho, o objetivo foi a implementação de controladores P, PI, PD, PID e PI-D de segunda ordem. Também foram adicionados ao controlador quatro medidas para análise do sistema de tanques: tempos de subida, pico e acomodação, além do sobre-sinal máximo do sistema. As equações teóricas utilizadas para formular o controlador estão explicitadas no texto deste relatório, bem como os algoritmos que simulam seus funcionamentos. O comportamento do sistema pode ser observado pelo usuário através de três gráficos presentes na interface do software, um para o sinal de entrada e os demais para os canais de saída.\\

\end{flushleft}

\vspace{1.5cm}

\textbf{Palavras-chave:} sistema de tanques; sistema de controle; software; planta Quanser, controlador PID.

\newpage

%%%%%%%%% LISTA DE FIGURAS %%%%%%%%%%%

\thispagestyle{empty}

\begin{center}
\listoffigures
\end{center}

\newpage

%%%%%%%%%%%%%%% SUMÁRIO %%%%%%%%%%%%%%

\thispagestyle{empty}

\begin{center}
\tableofcontents
\end{center}

\newpage

%%%%%%%%%%%%%%% INTRODUÇÃO %%%%%%%%%%%

\thispagestyle{main}

\section{INTRODUÇÃO}

\begin{flushleft}
\hspace{4ex}A prática de laboratório 04 tem como objetivo introduzir um sistema de controle em cascata na planta \textit{Quanser}, sendo a figura \ref{r2d2e} o simulador da planta:

\begin{figure}[H]
\centering
\includegraphics[width=11cm]{ImagensLab4/simulator.png}
\caption{R2D2E - Tank Simulator}
\label{r2d2e}
\end{figure}

\hspace{4ex}Pretende-se controlar o nível de água de ambos os tanques da planta, para isso, foi implementado os controladores Proporcional (P), Proporcional Integrativo (PI), Proporcional Derivativo (PD), Proporcional Integrativo Derivativo (PID) e Proporcional Integrativo Derivativo em controle de ação baseado no sinal do processo (PI-D). Quem irá escolher qual controle será usado é o usuário, através da interface do programa desenvolvido, como mostrado na figura \ref{interface}:

\begin{figure}[H]
\centering
\includegraphics[width=11cm]{ImagensLab4/interface-versao4.png}
\caption{Interface do software de controle}
\label{interface}
\end{figure}

\hspace{4ex}Além dessas opções, o usuário também deve escolher os ganhos dos controladores Kp, Ki e ou Kd, ou $\tau_i$ e $\tau_d$, os quais são necessários para o controlador, como também, o  sinal de referência a ser enviado, o offset, a análise da resposta do sistema para o tempo de subida (t$_r$), por exemplo, de 5\% à 95\%, e o tempo de acomodação (t$_s$), para as faixas de 2\%, 5\% e 10\% do degrau. Ademais, é possível saber os valores do  t$_r$, t$_s$, tempo de pico (t$_p$) e o sobressinal (Mp) do sistema de segunda ordem.

\hspace{4ex}O controle da planta se dará de duas formas: utilizando apenas uma malha de controle, como também, o controle em cascata. A configuração da planta mostrada na figura \ref{r2d2e} fazendo uso do controle de uma malha será da seguinte maneira:

\begin{figure}[H]
\centering
\includegraphics[width=11cm]{ImagensLab4/umamalha.png}
\caption{Configuração da planta utilizando uma malha de controle.}
\label{umamalha}
\end{figure}

\hspace{4ex}No caso mostrado acima, na figura \ref{umamalha}, a tensão enviada para a bomba está sendo manipulada para o tanque de baixo da planta alcançar o nível desejado pelo usuário, isso é possível por causa da relação entre os dois tanques: a vazão de saída do tanque de cima é a vazão de entrada do tanque de baixo. Através do cálculo do erro do sinal de referência e a medição do segundo tanque será fornecida uma tensão que deixe o tanque de cima com um nível de água a qual levará o tanque de baixo para o valor desejado.

\hspace{4ex}Com a configuração de duas malhas de controle, ou seja, utilizando o controle em cascata, teremos a seguinte configuração:

\begin{figure}[H]
\centering
\includegraphics[width=11cm]{ImagensLab4/duasmalhas.png}
\caption{Configuração da planta utilizando o controle em cascata.}
\label{duasmalha}
\end{figure}

\hspace{4ex}Nesse caso, além de controlar o nível do tanque de baixo, pode-se controlar o nível do tanque de cima. Nessa configuração, denominamos o controlador mais externo de controlador mestre, o qual controla o nível do tanque de baixo da mesma forma que ocorre no controle com uma malha, e o controlador da malha interna é o controlador escravo, que a partir do erro entre o nível desejado do tanque de cima e seu nível real, irá determinar a tensão enviada para a bomba.

\hspace{4ex}No relatório é apresentado a metodologia utilizada nessa experiência, os resultados obtidos através de testes e as conclusões a partir da comparação do comportamento do sistema com uma e com duas malhas.

\end{flushleft}

\newpage

%%%%%%%%%% REFERENCIAL TEÓRICO %%%%%%%

\thispagestyle{main}

\section{REFERENCIAL TEÓRICO}

\subsection{CONTROLE EM CASCATA}
\hspace{4ex}O controle em cascata é uma estratégia de controle que utiliza dois controladores aninhados de modo que a identificação de pequenos distúrbios aconteça de forma mais rápida e, consequentemente, a correção seja feita antes de o sistema sofrer grandes alterações. De forma genérica, o controle em cascata pode ser representando pelo diagrama da figura \ref{cascata}. Pode-se perceber que é necessário medir duas variáveis de processos, o da malha interna (chamada de malha escrava) e o da malha externa (chamada de malha mestre). Essas medidas, feitas através de uso de sensores, é comparada com a referência de cada malha. Sendo que a referência da malha externa, ou set point, é uma entrada do sistema. Já na malha interna, quem determina qual vai ser o "nível" a ser alcaçado é o controlador da malha externa. Então a malha do processo secundário recebe o erro, é feito o cálculo do nível para aquele processo que é enviado através de uma sinal de controle para o atuador, modificando o processo da malha interna.     

\begin{figure}[h]
\centering
\includegraphics[width=17cm]{ImagensLab4/controle-em-cascata.png}
\caption{Diagrama de bloco do controle em cascata}
\label{cascata}
\end{figure}

Esse tipo de configuração é eficiente devido a dependência entre o processo primário e o secundário. Por exemplo, numa planta onde o gás que passa por um cano é resposável pelo aquecimento da água de um tanque, a atuação é feita controlando a quantidade de vazão do gás no cano. Há uma relação nesse processo em que quanto maior a vazão, mais quente ficará a água. Então a malha mestre seria o tanque com a água, e a malha escrava, a tubulação com gás. 

Outro aspecto importante a ser levado em consideração é a velocidade de cada processo. O processo primário tem que ser mais lento que o processo secundário. Como a temperatura varia de forma mais lenta que a vazão de um gás por uma tubulação, então não haveria conflito entre os sinais de controle. 

A questão de utilizar dois controladores vem dessa diferença de velocidades entre os processos, visto que uma diferença pequena de temperatura geraria um pequeno sinal de erro, logo, se tivesse apenas um controlador, o sinal enviado para controlar a vazão seria pequeno, o que não surtiria muito efeito na temperatura, quase não alteraria ou demoraria muito tempo. Já dois controladores conseguem mapear a relaçao entre o processo lento e um processo mais rápido. 

Uma das desvantagens de se utilizar o controle em cascata é que, ao adicionar mais um controlador, aumenta a complexidade de controle, pois é necessário sintonizar um quantidade dobrada de ganhos, gerando uma quantidade maior de combinações entre tipos de controle e ganhos possíveis. 




\newpage

%%%%%%%%%% METODOLOGIA %%%%%%%%%%%%%%%

\thispagestyle{main}

\section{METODOLOGIA}

\hspace{4ex}Utilizando o simulador R2D2E, foram feitos testes com o controlador mestre e o escravo, como também, com apenas um controlador para alcançar determinados níveis dos tanques da planta, os quais serão expostos nos resultados.

\hspace{4ex}Para isso, foi implementado uma classe controlador a qual é determinada de acordo com as escolhas do usuário, ou seja, se é P, PI, PD, PID ou PI-D. Por exemplo, se o usuário escolhe o controlador PD, ele fornecerá apenas os valores de Kp e Kd, ou Kp e $\tau_d$, e a partir desses valores a classe assumirá que Ki ou $\tau_i$ não existe, ou seja, o valor da integral será zero. O mesmo acontece se for escolhido o PI, nesse caso, o valor da derivada do erro será zero.

\hspace{4ex}Observe os métodos abaixo, o primeiro é para os casos em que o usuário escolher os controladores P, PI, PD ou PID, enquanto que o segundo método foi criado para o caso em que se quer usar o PI-D, pois utiliza o sinal do processo (y) ao invés de usar o erro.
\begin{lstlisting}
double PID::Controle(double e, double h)//Controladores P, PI, PD e PID
{
    I= I+Ki*e*h; //Simpsons (e+e_ant)*h/2 - integral do erro
    D= Kd*(e-e_ant)/h; //Derivada do erro
    e_ant= e; //Erro
    return Kp*e+I+D; //Sinal de controle
}

double PID::Controle(double e, double y, double h)//Controlador PI-D
{
    I= I+Ki*e*h; //Integral do erro
    D= Kd*(y-e_ant)/h; //Derivada do sinal do processo
    e_ant= y; //Sinal do processo
    return Kp*e+I+D; //Sinal de controle
}
\end{lstlisting}
\hspace{4ex}No software desenvolvido, esses valores podem ser escolhidos como mostrado na figura \ref{controladores}.

\begin{figure}[H]
     \centering
     \subfloat[][Controlador Mestre]{\includegraphics[width=4cm]{ImagensLab4/controlador1}\label{<figureP1>}}\hspace{4ex}
     \subfloat[][Controlador Escravo]{\includegraphics[width=4cm]{ImagensLab4/controlador2}\label{<figureP2>}}\\
     \caption{Interface para a escolha dos controladores}
     \label{fig:controladores}
\end{figure}


\newpage

%%%%%%%%%% RESULTADOS %%%%%%%%%%%%%%%

\thispagestyle{main}

\section{RESULTADOS}

\hspace{4ex}A análise feita teve como foco a análise do desempenho do sistema de segunda ordem através de tempos específicos. Ou seja, o objetivo era controlar o segundo tanque do sistema, as leituras do nível aparecem no gráfico localizado inferior da janela.

\hspace{4ex}O primeiro teste foi feito utilizando o controle PI, na figura \ref{<figureP1>} se pode perceber o efeito do sobressinal negativo, onde se alcançou cerca de quase 18\% de  abaixo do nível estabelecido. Essa porcentagem é em relação a diferença do nível anterior para o nível configurado atualmente. No caso, alterou-se de 20 cm para 10 cm.

\hspace{4ex}Na figura \ref{<figureP2>}, nota-se claramente o tempo de subida de 12,3 s. Alguns valores como sobressinal e tempo de acomodação são atualizados constamente e mostrados durante o cálculo, então o valor real só extraído quando o sistema entra em regime permanente.
\begin{figure}[H]
     \centering
     \subfloat[][]{\includegraphics[width=10cm]{resultados-lab3/03}\label{<figureP1>}}\hspace{4ex}
     \subfloat[][]{\includegraphics[width=10cm]{resultados-lab3/04}\label{<figureP2>}}\\
     
     \caption{Análise do Controle PI}
     \subfloat[][]{\includegraphics[width=10cm]{resultados-lab3/04}\label{<figureP2>}}\\
     
     \caption{Análise do Controle PI}
     \label{fig:ControlePI}
\end{figure}

\hspace{4ex}Já no primeiro teste, notou-se que o controle do segundo tanque levou mais tempo para responder aos níveis de referência.

\hspace{4ex}Na \ref{<figurePID1>}, vê-se novamente o sobressinal agora positivo. O tempo de pico é calculado junto com o sobressinal e é mostrado logo acima do $M_p$ na parte de resultados do programa.

\hspace{4ex}Nas figuras \ref{<figurePID2>} e \ref{<figurePID3>}, mostra-se o tempo de acomodação obtido, 35 s. Foi considerado uma faixa de tolerância de 2\% que pode ser alterada pelo usuário. Percebe-se que os valores obtidos são uma boa aproximação, já que não se pode obter um tempo mais preciso devido à taxa de leitura dos sensores e resposta do sistema.
\begin{figure}[H]
     \centering
     \subfloat[][]{\includegraphics[width=10cm]{resultados-lab3/05}\label{<figurePID1>}}\hspace{4ex}
     \subfloat[][]{\includegraphics[width=10cm]{resultados-lab3/06}\label{<figurePID2>}}\\
     \subfloat[][]{\includegraphics[width=10cm]{resultados-lab3/07}\label{<figurePID3>}}
     
     \caption{Análise do Controle PID}
     \label{controlePID}
\end{figure}

\hspace{4ex} O efeito dos controladores no sistema continua o mesmo encontrado no sistema de primeira ordem. A oscilação do efeito derivativo e a resposta mais agressiva do efeito da integração. Sendo que como o sistema tem uma resposta mais lenta, faz-se necessário a análise dos tempos de subida, sobressinal e tempo de acomodação para os devidos ajustes nas constantes $K_p$, $K_D$ e $K_I$. Por exemplo, se o tempo de subida foi muito longo, é necessário o aumento do $K_I$ para uma resposta mais rápida. Mas devido aos efeitos de wind up do integrador e a amplificação do ruído no derivativo, o controlador precisa de filtros. No programa não foi implementado esse filtros, o que limita as faixas de uso das constantes.


\begin{figure}[H]
     \centering
     \includegraphics[width=10cm]{resultados-lab3/19}\label{<figurePD1>}
     \caption{Análise do Controle PD}
     \label{controlePD}
\end{figure}

Por fim o teste feito com o controlador PD. Foi usado uma constante $K_D$ baixa para evitar os efeitos negativos.



\newpage

%%%%%%%%%% CONCLUSÃO %%%%%%%%%%%%%%%

\thispagestyle{main}

\section{CONCLUSÃO}

\hspace{4ex}Os testes realizados num sistema de segunda ordem mostraram a importância da análise dos tempo de subida, tempo de pico, sobressinal e tempo de acomodação para o controle de um sistema. Estes valores obtidos experimentalmente podem ser aplicados no modelo matemático, para então encontrar a função que rege o sistema podendo assim ser mais fácil sintonizar o controlador e projetar. Como foi utilizado teorias aplicadas para sistemas lineares, essas soluções podem ser generalizadas e adaptadas para qualquer sistema linear. 
\newpage

%%%%%%%% REFERÊNCIAS %%%%%%%%%%%%%%%%%

\thispagestyle{empty}
\section{BIBLIOGRAFIA}

Fundamentals of cascade control | Control Engineering. Disponível em: <http://www.controleng.com/single-article/fundamentals-of-cascade-control/bcedad6518aec409f583ba6bc9b72854.html>. Acesso em: 10 maio. 2017.




%Referências bibliogáficas (geradas automaticamente)

%\addcontentsline{toc}{chapter}{Referências bibliográficas}
%\bibliography{bib/bibliografia}

%\appendix

%Apêndice A
%\include{apendice}

\end{document}
